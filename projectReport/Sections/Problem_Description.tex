%! Author = maximilian.j.mueller
%! Date = 13.01.2025
\clearpage
\section{Problem Description}

The customer is the operator of electric vehicle (EV) charging hubs.
EVs provide a substantial improvement of individual carbon footprint and a switch from an internal combustion engine vehicle (ICE) to EVs can help reduce the second-largest chunk of total EU emissions.
The understanding and usage of EVs is still limited, mostly due to personal misunderstandings or unfamiliarity with the technology.
Additionally, EVs are used and charged fundamentally different from ICEs as users wish to charge the EV while it is parked or not in use, whether that might be at home or at a public location like a shopping mall, compared to ICEs which are quickly refilled at a gas station.
This adds strain on the power grid that needs to be accounted for when planning and building charging infrastructure, both private and public.
In order to efficiently and successfully plan future charging infrastructure, the operator of charging hubs has tasked us with better understanding of the systems metrics as well as its utilization prediction